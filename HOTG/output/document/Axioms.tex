%
\begin{isabellebody}%
\setisabellecontext{Axioms}%
%
\isadelimdocument
%
\endisadelimdocument
%
\isatagdocument
\isanewline
%
\isamarkupsection{Axioms of Tarski-Grothendieck Set Theory embedded in HOL.%
}
\isamarkuptrue%
%
\endisatagdocument
{\isafolddocument}%
%
\isadelimdocument
%
\endisadelimdocument
%
\isadelimtheory
%
\endisadelimtheory
%
\isatagtheory
\isacommand{theory}\isamarkupfalse%
\ Axioms\isanewline
\ \ \isakeyword{imports}\ Setup\isanewline
\isakeyword{begin}%
\endisatagtheory
{\isafoldtheory}%
%
\isadelimtheory
%
\endisadelimtheory
%
\isadelimdocument
%
\endisadelimdocument
%
\isatagdocument
%
\isamarkupparagraph{Summary%
}
\isamarkuptrue%
%
\endisatagdocument
{\isafolddocument}%
%
\isadelimdocument
%
\endisadelimdocument
%
\begin{isamarkuptext}%
We follow the axiomatisation as described in \cite{brown_et_al:LIPIcs:2019:11064},
who also describe the existence of a model if a 2-inaccessible cardinal exists.%
\end{isamarkuptext}\isamarkuptrue%
%
\begin{isamarkuptext}%
The primitive set type.%
\end{isamarkuptext}\isamarkuptrue%
\isacommand{typedecl}\isamarkupfalse%
\ set%
\begin{isamarkuptext}%
The first four axioms.%
\end{isamarkuptext}\isamarkuptrue%
\isacommand{axiomatization}\isamarkupfalse%
\isanewline
\ \ mem\ \ \ \ \ \ {\isacharcolon}{\kern0pt}{\isacharcolon}{\kern0pt}\ {\isacartoucheopen}set\ {\isasymRightarrow}\ set\ {\isasymRightarrow}\ bool{\isacartoucheclose}\ \isakeyword{and}\isanewline
\ \ emptyset\ {\isacharcolon}{\kern0pt}{\isacharcolon}{\kern0pt}\ {\isacartoucheopen}set{\isacartoucheclose}\ \isakeyword{and}\isanewline
\ \ union\ \ \ \ \ \ \ {\isacharcolon}{\kern0pt}{\isacharcolon}{\kern0pt}\ {\isacartoucheopen}set\ {\isasymRightarrow}\ set{\isacartoucheclose}\ \isakeyword{and}\isanewline
\ \ repl\ \ \ \ \ {\isacharcolon}{\kern0pt}{\isacharcolon}{\kern0pt}\ {\isacartoucheopen}set\ {\isasymRightarrow}\ {\isacharparenleft}{\kern0pt}set\ {\isasymRightarrow}\ set{\isacharparenright}{\kern0pt}\ {\isasymRightarrow}\ set{\isacartoucheclose}\isanewline
\isakeyword{where}\isanewline
\ \ mem{\isacharunderscore}{\kern0pt}induction{\isacharcolon}{\kern0pt}\ {\isachardoublequoteopen}{\isacharparenleft}{\kern0pt}{\isasymforall}X{\isachardot}{\kern0pt}\ {\isacharparenleft}{\kern0pt}{\isasymforall}x{\isachardot}{\kern0pt}\ mem\ x\ X\ {\isasymlongrightarrow}\ P\ x{\isacharparenright}{\kern0pt}\ {\isasymlongrightarrow}\ P\ X{\isacharparenright}{\kern0pt}\ {\isasymlongrightarrow}\ {\isacharparenleft}{\kern0pt}{\isasymforall}X{\isachardot}{\kern0pt}\ P\ X{\isacharparenright}{\kern0pt}{\isachardoublequoteclose}\ \isakeyword{and}\isanewline
\ \ emptyset{\isacharcolon}{\kern0pt}\ {\isachardoublequoteopen}{\isasymnot}{\isacharparenleft}{\kern0pt}{\isasymexists}x{\isachardot}{\kern0pt}\ mem\ x\ emptyset{\isacharparenright}{\kern0pt}{\isachardoublequoteclose}\ \isakeyword{and}\isanewline
\ \ union{\isacharcolon}{\kern0pt}\ {\isachardoublequoteopen}{\isasymforall}X\ x{\isachardot}{\kern0pt}\ mem\ x\ {\isacharparenleft}{\kern0pt}union\ X{\isacharparenright}{\kern0pt}\ {\isasymlongleftrightarrow}\ {\isacharparenleft}{\kern0pt}{\isasymexists}Y{\isachardot}{\kern0pt}\ mem\ Y\ X\ {\isasymand}\ mem\ x\ Y{\isacharparenright}{\kern0pt}{\isachardoublequoteclose}\ \isakeyword{and}\isanewline
\ \ replacement{\isacharcolon}{\kern0pt}\ {\isachardoublequoteopen}{\isasymforall}X\ y{\isachardot}{\kern0pt}\ mem\ y\ {\isacharparenleft}{\kern0pt}repl\ X\ f{\isacharparenright}{\kern0pt}\ {\isasymlongleftrightarrow}\ {\isacharparenleft}{\kern0pt}{\isasymexists}x{\isachardot}{\kern0pt}\ mem\ x\ X\ {\isasymand}\ y\ {\isacharequal}{\kern0pt}\ f\ x{\isacharparenright}{\kern0pt}{\isachardoublequoteclose}%
\begin{isamarkuptext}%
Note: axioms \isa{{\isacharparenleft}{\kern0pt}{\isasymforall}X{\isachardot}{\kern0pt}\ {\isacharparenleft}{\kern0pt}{\isasymforall}x{\isachardot}{\kern0pt}\ mem\ x\ X\ {\isasymlongrightarrow}\ {\isacharquery}{\kern0pt}P\ x{\isacharparenright}{\kern0pt}\ {\isasymlongrightarrow}\ {\isacharquery}{\kern0pt}P\ X{\isacharparenright}{\kern0pt}\ {\isasymlongrightarrow}\ {\isacharparenleft}{\kern0pt}{\isasymforall}X{\isachardot}{\kern0pt}\ {\isacharquery}{\kern0pt}P\ X{\isacharparenright}{\kern0pt}} and \isa{{\isasymforall}X\ y{\isachardot}{\kern0pt}\ mem\ y\ {\isacharparenleft}{\kern0pt}repl\ X\ {\isacharquery}{\kern0pt}f{\isacharparenright}{\kern0pt}\ {\isacharequal}{\kern0pt}\ {\isacharparenleft}{\kern0pt}{\isasymexists}x{\isachardot}{\kern0pt}\ mem\ x\ X\ {\isasymand}\ y\ {\isacharequal}{\kern0pt}\ {\isacharquery}{\kern0pt}f\ x{\isacharparenright}{\kern0pt}} are axiom schemas
in first-order logic. Moreover, \isa{{\isasymforall}X\ y{\isachardot}{\kern0pt}\ mem\ y\ {\isacharparenleft}{\kern0pt}repl\ X\ {\isacharquery}{\kern0pt}f{\isacharparenright}{\kern0pt}\ {\isacharequal}{\kern0pt}\ {\isacharparenleft}{\kern0pt}{\isasymexists}x{\isachardot}{\kern0pt}\ mem\ x\ X\ {\isasymand}\ y\ {\isacharequal}{\kern0pt}\ {\isacharquery}{\kern0pt}f\ x{\isacharparenright}{\kern0pt}} takes a meta-level function \isa{F}.%
\end{isamarkuptext}\isamarkuptrue%
%
\begin{isamarkuptext}%
Let us define some expected notation.%
\end{isamarkuptext}\isamarkuptrue%
\isacommand{bundle}\isamarkupfalse%
\ hotg{\isacharunderscore}{\kern0pt}mem{\isacharunderscore}{\kern0pt}syntax\ \isakeyword{begin}\ \isacommand{notation}\isamarkupfalse%
\ mem\ {\isacharparenleft}{\kern0pt}\isakeyword{infixl}\ {\isachardoublequoteopen}{\isasymin}{\isachardoublequoteclose}\ {\isadigit{5}}{\isadigit{0}}{\isacharparenright}{\kern0pt}\ \isacommand{end}\isamarkupfalse%
\isanewline
\isacommand{bundle}\isamarkupfalse%
\ no{\isacharunderscore}{\kern0pt}hotg{\isacharunderscore}{\kern0pt}mem{\isacharunderscore}{\kern0pt}syntax\ \isakeyword{begin}\ \isacommand{no{\isacharunderscore}{\kern0pt}notation}\isamarkupfalse%
\ mem\ {\isacharparenleft}{\kern0pt}\isakeyword{infixl}\ {\isachardoublequoteopen}{\isasymin}{\isachardoublequoteclose}\ {\isadigit{5}}{\isadigit{0}}{\isacharparenright}{\kern0pt}\ \isacommand{end}\isamarkupfalse%
\isanewline
\isanewline
\isacommand{bundle}\isamarkupfalse%
\ hotg{\isacharunderscore}{\kern0pt}emptyset{\isacharunderscore}{\kern0pt}zero{\isacharunderscore}{\kern0pt}syntax\ \isakeyword{begin}\ \isacommand{notation}\isamarkupfalse%
\ emptyset\ {\isacharparenleft}{\kern0pt}{\isachardoublequoteopen}{\isasymemptyset}{\isachardoublequoteclose}{\isacharparenright}{\kern0pt}\ \isacommand{end}\isamarkupfalse%
\isanewline
\isacommand{bundle}\isamarkupfalse%
\ no{\isacharunderscore}{\kern0pt}hotg{\isacharunderscore}{\kern0pt}emptyset{\isacharunderscore}{\kern0pt}zero{\isacharunderscore}{\kern0pt}syntax\ \isakeyword{begin}\ \isacommand{no{\isacharunderscore}{\kern0pt}notation}\isamarkupfalse%
\ emptyset\ {\isacharparenleft}{\kern0pt}{\isachardoublequoteopen}{\isasymemptyset}{\isachardoublequoteclose}{\isacharparenright}{\kern0pt}\ \isacommand{end}\isamarkupfalse%
\isanewline
\isanewline
\isacommand{bundle}\isamarkupfalse%
\ hotg{\isacharunderscore}{\kern0pt}emptyset{\isacharunderscore}{\kern0pt}braces{\isacharunderscore}{\kern0pt}syntax\ \isakeyword{begin}\ \isacommand{notation}\isamarkupfalse%
\ emptyset\ {\isacharparenleft}{\kern0pt}{\isachardoublequoteopen}{\isacharbraceleft}{\kern0pt}{\isacharbraceright}{\kern0pt}{\isachardoublequoteclose}{\isacharparenright}{\kern0pt}\ \isacommand{end}\isamarkupfalse%
\isanewline
\isacommand{bundle}\isamarkupfalse%
\ no{\isacharunderscore}{\kern0pt}hotg{\isacharunderscore}{\kern0pt}emptyset{\isacharunderscore}{\kern0pt}braces{\isacharunderscore}{\kern0pt}syntax\ \isakeyword{begin}\ \isacommand{no{\isacharunderscore}{\kern0pt}notation}\isamarkupfalse%
\ emptyset\ {\isacharparenleft}{\kern0pt}{\isachardoublequoteopen}{\isacharbraceleft}{\kern0pt}{\isacharbraceright}{\kern0pt}{\isachardoublequoteclose}{\isacharparenright}{\kern0pt}\ \isacommand{end}\isamarkupfalse%
\isanewline
\isanewline
\isacommand{bundle}\isamarkupfalse%
\ hotg{\isacharunderscore}{\kern0pt}emptyset{\isacharunderscore}{\kern0pt}syntax\isanewline
\isakeyword{begin}\isanewline
\ \ \isacommand{unbundle}\isamarkupfalse%
\ hotg{\isacharunderscore}{\kern0pt}emptyset{\isacharunderscore}{\kern0pt}zero{\isacharunderscore}{\kern0pt}syntax\ hotg{\isacharunderscore}{\kern0pt}emptyset{\isacharunderscore}{\kern0pt}braces{\isacharunderscore}{\kern0pt}syntax\isanewline
\isacommand{end}\isamarkupfalse%
\isanewline
\isacommand{bundle}\isamarkupfalse%
\ no{\isacharunderscore}{\kern0pt}hotg{\isacharunderscore}{\kern0pt}emptyset{\isacharunderscore}{\kern0pt}syntax\isanewline
\isakeyword{begin}\isanewline
\ \ \isacommand{unbundle}\isamarkupfalse%
\ no{\isacharunderscore}{\kern0pt}hotg{\isacharunderscore}{\kern0pt}emptyset{\isacharunderscore}{\kern0pt}zero{\isacharunderscore}{\kern0pt}syntax\ no{\isacharunderscore}{\kern0pt}hotg{\isacharunderscore}{\kern0pt}emptyset{\isacharunderscore}{\kern0pt}braces{\isacharunderscore}{\kern0pt}syntax\isanewline
\isacommand{end}\isamarkupfalse%
\isanewline
\isanewline
\isacommand{bundle}\isamarkupfalse%
\ hotg{\isacharunderscore}{\kern0pt}union{\isacharunderscore}{\kern0pt}syntax\ \isakeyword{begin}\ \isacommand{notation}\isamarkupfalse%
\ union\ {\isacharparenleft}{\kern0pt}{\isachardoublequoteopen}{\isasymUnion}{\isacharunderscore}{\kern0pt}{\isachardoublequoteclose}\ {\isacharbrackleft}{\kern0pt}{\isadigit{9}}{\isadigit{0}}{\isacharbrackright}{\kern0pt}\ {\isadigit{9}}{\isadigit{0}}{\isacharparenright}{\kern0pt}\ \isacommand{end}\isamarkupfalse%
\isanewline
\isacommand{bundle}\isamarkupfalse%
\ no{\isacharunderscore}{\kern0pt}hotg{\isacharunderscore}{\kern0pt}union{\isacharunderscore}{\kern0pt}syntax\ \isakeyword{begin}\ \isacommand{no{\isacharunderscore}{\kern0pt}notation}\isamarkupfalse%
\ union\ {\isacharparenleft}{\kern0pt}{\isachardoublequoteopen}{\isasymUnion}{\isacharunderscore}{\kern0pt}{\isachardoublequoteclose}\ {\isacharbrackleft}{\kern0pt}{\isadigit{9}}{\isadigit{0}}{\isacharbrackright}{\kern0pt}\ {\isadigit{9}}{\isadigit{0}}{\isacharparenright}{\kern0pt}\ \isacommand{end}\isamarkupfalse%
\isanewline
\isanewline
\isacommand{unbundle}\isamarkupfalse%
\ hotg{\isacharunderscore}{\kern0pt}mem{\isacharunderscore}{\kern0pt}syntax\ hotg{\isacharunderscore}{\kern0pt}emptyset{\isacharunderscore}{\kern0pt}syntax\ hotg{\isacharunderscore}{\kern0pt}union{\isacharunderscore}{\kern0pt}syntax\isanewline
\isanewline
\isacommand{abbreviation}\isamarkupfalse%
\ {\isacharparenleft}{\kern0pt}input{\isacharparenright}{\kern0pt}\ {\isachardoublequoteopen}mem{\isacharunderscore}{\kern0pt}of\ A\ x\ {\isasymequiv}\ x\ {\isasymin}\ A{\isachardoublequoteclose}\isanewline
\isacommand{abbreviation}\isamarkupfalse%
\ {\isachardoublequoteopen}not{\isacharunderscore}{\kern0pt}mem\ x\ y\ {\isasymequiv}\ {\isasymnot}{\isacharparenleft}{\kern0pt}x\ {\isasymin}\ y{\isacharparenright}{\kern0pt}{\isachardoublequoteclose}\isanewline
\isanewline
\isacommand{bundle}\isamarkupfalse%
\ hotg{\isacharunderscore}{\kern0pt}not{\isacharunderscore}{\kern0pt}mem{\isacharunderscore}{\kern0pt}syntax\ \isakeyword{begin}\ \isacommand{notation}\isamarkupfalse%
\ not{\isacharunderscore}{\kern0pt}mem\ {\isacharparenleft}{\kern0pt}\isakeyword{infixl}\ {\isachardoublequoteopen}{\isasymnotin}{\isachardoublequoteclose}\ {\isadigit{5}}{\isadigit{0}}{\isacharparenright}{\kern0pt}\ \isacommand{end}\isamarkupfalse%
\isanewline
\isacommand{bundle}\isamarkupfalse%
\ no{\isacharunderscore}{\kern0pt}hotg{\isacharunderscore}{\kern0pt}not{\isacharunderscore}{\kern0pt}mem{\isacharunderscore}{\kern0pt}syntax\ \isakeyword{begin}\ \isacommand{no{\isacharunderscore}{\kern0pt}notation}\isamarkupfalse%
\ not{\isacharunderscore}{\kern0pt}mem\ {\isacharparenleft}{\kern0pt}\isakeyword{infixl}\ {\isachardoublequoteopen}{\isasymnotin}{\isachardoublequoteclose}\ {\isadigit{5}}{\isadigit{0}}{\isacharparenright}{\kern0pt}\ \isacommand{end}\isamarkupfalse%
\isanewline
\isanewline
\isacommand{unbundle}\isamarkupfalse%
\ hotg{\isacharunderscore}{\kern0pt}not{\isacharunderscore}{\kern0pt}mem{\isacharunderscore}{\kern0pt}syntax%
\begin{isamarkuptext}%
Based on the membership relation, we can define the subset relation.%
\end{isamarkuptext}\isamarkuptrue%
\isacommand{definition}\isamarkupfalse%
\ subset\ {\isacharcolon}{\kern0pt}{\isacharcolon}{\kern0pt}\ {\isacartoucheopen}set\ {\isasymRightarrow}\ set\ {\isasymRightarrow}\ bool{\isacartoucheclose}\isanewline
\ \ \isakeyword{where}\ {\isachardoublequoteopen}subset\ A\ B\ {\isasymequiv}\ {\isasymforall}x{\isachardot}{\kern0pt}\ x\ {\isasymin}\ A\ {\isasymlongrightarrow}\ x\ {\isasymin}\ B{\isachardoublequoteclose}%
\begin{isamarkuptext}%
Again, we define some notation.%
\end{isamarkuptext}\isamarkuptrue%
\isacommand{bundle}\isamarkupfalse%
\ hotg{\isacharunderscore}{\kern0pt}subset{\isacharunderscore}{\kern0pt}syntax\ \isakeyword{begin}\ \isacommand{notation}\isamarkupfalse%
\ subset\ {\isacharparenleft}{\kern0pt}\isakeyword{infixl}\ {\isachardoublequoteopen}{\isasymsubseteq}{\isachardoublequoteclose}\ {\isadigit{5}}{\isadigit{0}}{\isacharparenright}{\kern0pt}\ \isacommand{end}\isamarkupfalse%
\isanewline
\isacommand{bundle}\isamarkupfalse%
\ no{\isacharunderscore}{\kern0pt}hotg{\isacharunderscore}{\kern0pt}subset{\isacharunderscore}{\kern0pt}syntax\ \isakeyword{begin}\ \isacommand{no{\isacharunderscore}{\kern0pt}notation}\isamarkupfalse%
\ subset\ {\isacharparenleft}{\kern0pt}\isakeyword{infixl}\ {\isachardoublequoteopen}{\isasymsubseteq}{\isachardoublequoteclose}\ {\isadigit{5}}{\isadigit{0}}{\isacharparenright}{\kern0pt}\ \isacommand{end}\isamarkupfalse%
\isanewline
\isanewline
\isacommand{unbundle}\isamarkupfalse%
\ hotg{\isacharunderscore}{\kern0pt}subset{\isacharunderscore}{\kern0pt}syntax%
\begin{isamarkuptext}%
The axiom of extensionality and powerset.%
\end{isamarkuptext}\isamarkuptrue%
\isacommand{axiomatization}\isamarkupfalse%
\isanewline
\ \ powerset\ {\isacharcolon}{\kern0pt}{\isacharcolon}{\kern0pt}\ {\isacartoucheopen}set\ {\isasymRightarrow}\ set{\isacartoucheclose}\isanewline
\isakeyword{where}\isanewline
\ \ extensionality{\isacharcolon}{\kern0pt}\ {\isachardoublequoteopen}{\isasymforall}X\ Y{\isachardot}{\kern0pt}\ X\ {\isasymsubseteq}\ Y\ {\isasymlongrightarrow}\ Y\ {\isasymsubseteq}\ X\ {\isasymlongrightarrow}\ X\ {\isacharequal}{\kern0pt}\ Y{\isachardoublequoteclose}\ \isakeyword{and}\isanewline
\ \ powerset{\isacharcolon}{\kern0pt}\ {\isachardoublequoteopen}{\isasymforall}A\ x{\isachardot}{\kern0pt}\ x\ {\isasymin}\ powerset\ A\ {\isasymlongleftrightarrow}\ x\ {\isasymsubseteq}\ A{\isachardoublequoteclose}%
\begin{isamarkuptext}%
Lastly, we want to axiomatise the existence of Grothendieck universes.
This can be done in different ways. We again follow the approach from
\cite{brown_et_al:LIPIcs:2019:11064}.%
\end{isamarkuptext}\isamarkuptrue%
\isacommand{definition}\isamarkupfalse%
\ mem{\isacharunderscore}{\kern0pt}trans{\isacharunderscore}{\kern0pt}closed\ {\isacharcolon}{\kern0pt}{\isacharcolon}{\kern0pt}\ {\isacartoucheopen}set\ {\isasymRightarrow}\ bool{\isacartoucheclose}\isanewline
\ \ \isakeyword{where}\ {\isachardoublequoteopen}mem{\isacharunderscore}{\kern0pt}trans{\isacharunderscore}{\kern0pt}closed\ X\ {\isasymequiv}\ {\isacharparenleft}{\kern0pt}{\isasymforall}x{\isachardot}{\kern0pt}\ x\ {\isasymin}\ X\ {\isasymlongrightarrow}\ x\ {\isasymsubseteq}\ X{\isacharparenright}{\kern0pt}{\isachardoublequoteclose}\isanewline
\isanewline
\isacommand{definition}\isamarkupfalse%
\ ZF{\isacharunderscore}{\kern0pt}closed\ {\isacharcolon}{\kern0pt}{\isacharcolon}{\kern0pt}\ {\isacartoucheopen}set\ {\isasymRightarrow}\ bool{\isacartoucheclose}\isanewline
\ \ \isakeyword{where}\ {\isachardoublequoteopen}ZF{\isacharunderscore}{\kern0pt}closed\ U\ {\isasymequiv}\ {\isacharparenleft}{\kern0pt}\isanewline
\ \ \ \ {\isacharparenleft}{\kern0pt}{\isasymforall}X{\isachardot}{\kern0pt}\ X\ {\isasymin}\ U\ {\isasymlongrightarrow}\ {\isasymUnion}X\ {\isasymin}\ U{\isacharparenright}{\kern0pt}\ {\isasymand}\isanewline
\ \ \ \ {\isacharparenleft}{\kern0pt}{\isasymforall}X{\isachardot}{\kern0pt}\ X\ {\isasymin}\ U\ {\isasymlongrightarrow}\ powerset\ X\ {\isasymin}\ U{\isacharparenright}{\kern0pt}\ {\isasymand}\isanewline
\ \ \ \ {\isacharparenleft}{\kern0pt}{\isasymforall}X\ F{\isachardot}{\kern0pt}\ X\ {\isasymin}\ U\ {\isasymlongrightarrow}\ {\isacharparenleft}{\kern0pt}{\isasymforall}x{\isachardot}{\kern0pt}\ x\ {\isasymin}\ X\ {\isasymlongrightarrow}\ F\ x\ {\isasymin}\ U{\isacharparenright}{\kern0pt}\ {\isasymlongrightarrow}\ repl\ X\ F\ {\isasymin}\ U{\isacharparenright}{\kern0pt}\isanewline
\ \ {\isacharparenright}{\kern0pt}{\isachardoublequoteclose}%
\begin{isamarkuptext}%
Note that \isa{ZF{\isacharunderscore}{\kern0pt}closed} is a second-order statement.%
\end{isamarkuptext}\isamarkuptrue%
%
\begin{isamarkuptext}%
\isa{univ\ X} is the smallest Grothendieck universe containing X.%
\end{isamarkuptext}\isamarkuptrue%
\isacommand{axiomatization}\isamarkupfalse%
\isanewline
\ \ univ\ {\isacharcolon}{\kern0pt}{\isacharcolon}{\kern0pt}\ {\isacartoucheopen}set\ {\isasymRightarrow}\ set{\isacartoucheclose}\isanewline
\isakeyword{where}\isanewline
\ \ mem{\isacharunderscore}{\kern0pt}univ\ {\isacharbrackleft}{\kern0pt}iff{\isacharbrackright}{\kern0pt}{\isacharcolon}{\kern0pt}\ {\isachardoublequoteopen}X\ {\isasymin}\ univ\ X{\isachardoublequoteclose}\ \isakeyword{and}\isanewline
\ \ mem{\isacharunderscore}{\kern0pt}trans{\isacharunderscore}{\kern0pt}closed{\isacharunderscore}{\kern0pt}univ\ {\isacharbrackleft}{\kern0pt}iff{\isacharbrackright}{\kern0pt}{\isacharcolon}{\kern0pt}\ {\isachardoublequoteopen}mem{\isacharunderscore}{\kern0pt}trans{\isacharunderscore}{\kern0pt}closed\ {\isacharparenleft}{\kern0pt}univ\ X{\isacharparenright}{\kern0pt}{\isachardoublequoteclose}\ \isakeyword{and}\isanewline
\ \ ZF{\isacharunderscore}{\kern0pt}closed{\isacharunderscore}{\kern0pt}univ\ {\isacharbrackleft}{\kern0pt}iff{\isacharbrackright}{\kern0pt}{\isacharcolon}{\kern0pt}\ {\isachardoublequoteopen}ZF{\isacharunderscore}{\kern0pt}closed\ {\isacharparenleft}{\kern0pt}univ\ X{\isacharparenright}{\kern0pt}{\isachardoublequoteclose}\ \isakeyword{and}\isanewline
\ \ univ{\isacharunderscore}{\kern0pt}min{\isacharcolon}{\kern0pt}\ {\isachardoublequoteopen}{\isasymlbrakk}X\ {\isasymin}\ U{\isacharsemicolon}{\kern0pt}\ mem{\isacharunderscore}{\kern0pt}trans{\isacharunderscore}{\kern0pt}closed\ U{\isacharsemicolon}{\kern0pt}\ ZF{\isacharunderscore}{\kern0pt}closed\ U{\isasymrbrakk}\ {\isasymLongrightarrow}\ univ\ X\ {\isasymsubseteq}\ U{\isachardoublequoteclose}\isanewline
\isanewline
\isanewline
\isacommand{bundle}\isamarkupfalse%
\ hotg{\isacharunderscore}{\kern0pt}basic{\isacharunderscore}{\kern0pt}syntax\isanewline
\isakeyword{begin}\isanewline
\ \ \isacommand{unbundle}\isamarkupfalse%
\isanewline
\ \ \ \ hotg{\isacharunderscore}{\kern0pt}mem{\isacharunderscore}{\kern0pt}syntax\isanewline
\ \ \ \ hotg{\isacharunderscore}{\kern0pt}not{\isacharunderscore}{\kern0pt}mem{\isacharunderscore}{\kern0pt}syntax\isanewline
\ \ \ \ hotg{\isacharunderscore}{\kern0pt}emptyset{\isacharunderscore}{\kern0pt}syntax\isanewline
\ \ \ \ hotg{\isacharunderscore}{\kern0pt}union{\isacharunderscore}{\kern0pt}syntax\isanewline
\ \ \ \ hotg{\isacharunderscore}{\kern0pt}subset{\isacharunderscore}{\kern0pt}syntax\isanewline
\isacommand{end}\isamarkupfalse%
\isanewline
\isacommand{bundle}\isamarkupfalse%
\ no{\isacharunderscore}{\kern0pt}hotg{\isacharunderscore}{\kern0pt}basic{\isacharunderscore}{\kern0pt}syntax\isanewline
\isakeyword{begin}\isanewline
\ \ \isacommand{unbundle}\isamarkupfalse%
\isanewline
\ \ \ \ no{\isacharunderscore}{\kern0pt}hotg{\isacharunderscore}{\kern0pt}mem{\isacharunderscore}{\kern0pt}syntax\isanewline
\ \ \ \ no{\isacharunderscore}{\kern0pt}hotg{\isacharunderscore}{\kern0pt}not{\isacharunderscore}{\kern0pt}mem{\isacharunderscore}{\kern0pt}syntax\isanewline
\ \ \ \ no{\isacharunderscore}{\kern0pt}hotg{\isacharunderscore}{\kern0pt}emptyset{\isacharunderscore}{\kern0pt}syntax\isanewline
\ \ \ \ no{\isacharunderscore}{\kern0pt}hotg{\isacharunderscore}{\kern0pt}union{\isacharunderscore}{\kern0pt}syntax\isanewline
\ \ \ \ no{\isacharunderscore}{\kern0pt}hotg{\isacharunderscore}{\kern0pt}subset{\isacharunderscore}{\kern0pt}syntax\isanewline
\isacommand{end}\isamarkupfalse%
\isanewline
%
\isadelimtheory
\isanewline
%
\endisadelimtheory
%
\isatagtheory
\isacommand{end}\isamarkupfalse%
%
\endisatagtheory
{\isafoldtheory}%
%
\isadelimtheory
%
\endisadelimtheory
%
\end{isabellebody}%
\endinput
%:%file=~/Documents/github/Isabelle-Set/HOTG/Axioms.thy%:%
%:%10=1%:%
%:%12=3%:%
%:%28=4%:%
%:%29=4%:%
%:%30=5%:%
%:%31=6%:%
%:%45=7%:%
%:%57=8%:%
%:%58=9%:%
%:%62=11%:%
%:%64=12%:%
%:%65=12%:%
%:%67=14%:%
%:%69=15%:%
%:%70=15%:%
%:%71=16%:%
%:%72=17%:%
%:%73=18%:%
%:%74=19%:%
%:%75=20%:%
%:%76=21%:%
%:%77=22%:%
%:%78=23%:%
%:%79=24%:%
%:%81=26%:%
%:%82=27%:%
%:%86=29%:%
%:%88=31%:%
%:%89=31%:%
%:%90=31%:%
%:%91=31%:%
%:%92=32%:%
%:%93=32%:%
%:%94=32%:%
%:%95=32%:%
%:%96=33%:%
%:%97=34%:%
%:%98=34%:%
%:%99=34%:%
%:%100=34%:%
%:%101=35%:%
%:%102=35%:%
%:%103=35%:%
%:%104=35%:%
%:%105=36%:%
%:%106=37%:%
%:%107=37%:%
%:%108=37%:%
%:%109=37%:%
%:%110=38%:%
%:%111=38%:%
%:%112=38%:%
%:%113=38%:%
%:%114=39%:%
%:%115=40%:%
%:%116=40%:%
%:%117=41%:%
%:%118=42%:%
%:%119=42%:%
%:%120=43%:%
%:%121=43%:%
%:%122=44%:%
%:%123=44%:%
%:%124=45%:%
%:%125=46%:%
%:%126=46%:%
%:%127=47%:%
%:%128=47%:%
%:%129=48%:%
%:%130=49%:%
%:%131=49%:%
%:%132=49%:%
%:%133=49%:%
%:%134=50%:%
%:%135=50%:%
%:%136=50%:%
%:%137=50%:%
%:%138=51%:%
%:%139=52%:%
%:%140=52%:%
%:%141=53%:%
%:%142=54%:%
%:%143=54%:%
%:%144=55%:%
%:%145=55%:%
%:%146=56%:%
%:%147=57%:%
%:%148=57%:%
%:%149=57%:%
%:%150=57%:%
%:%151=58%:%
%:%152=58%:%
%:%153=58%:%
%:%154=58%:%
%:%155=59%:%
%:%156=60%:%
%:%157=60%:%
%:%159=63%:%
%:%161=64%:%
%:%162=64%:%
%:%163=65%:%
%:%165=67%:%
%:%167=68%:%
%:%168=68%:%
%:%169=68%:%
%:%170=68%:%
%:%171=69%:%
%:%172=69%:%
%:%173=69%:%
%:%174=69%:%
%:%175=70%:%
%:%176=71%:%
%:%177=71%:%
%:%179=73%:%
%:%181=74%:%
%:%182=74%:%
%:%183=75%:%
%:%184=76%:%
%:%185=77%:%
%:%186=78%:%
%:%188=80%:%
%:%189=81%:%
%:%190=82%:%
%:%192=84%:%
%:%193=84%:%
%:%194=85%:%
%:%195=86%:%
%:%196=87%:%
%:%197=87%:%
%:%198=88%:%
%:%204=94%:%
%:%208=96%:%
%:%210=98%:%
%:%211=98%:%
%:%212=99%:%
%:%213=100%:%
%:%214=101%:%
%:%215=102%:%
%:%216=103%:%
%:%217=104%:%
%:%218=105%:%
%:%219=106%:%
%:%220=107%:%
%:%221=107%:%
%:%222=108%:%
%:%223=109%:%
%:%224=109%:%
%:%225=110%:%
%:%226=111%:%
%:%227=112%:%
%:%228=113%:%
%:%229=114%:%
%:%230=115%:%
%:%231=115%:%
%:%232=116%:%
%:%233=116%:%
%:%234=117%:%
%:%235=118%:%
%:%236=118%:%
%:%237=119%:%
%:%238=120%:%
%:%239=121%:%
%:%240=122%:%
%:%241=123%:%
%:%242=124%:%
%:%243=124%:%
%:%246=125%:%
%:%251=126%:%
